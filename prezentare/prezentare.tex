\documentclass{beamer}
\usetheme{Copenhagen}
\usecolortheme{default}
%Information to be included in the title page:
\title[Gemixque - sistem de recomandări de jocuri video]{Gemixque}
\subtitle{Sistem de recomandări de jocuri video}
\author[Radu Damian]{Radu Damian \and \\[9mm] Dr. Cristian Frăsinaru}

\institute{Facultatea de Informatică}
\date{2022}

\setbeamertemplate{headline}{}
\begin{document}

\frame{\titlepage}


\begin{frame}
  \frametitle{Cuprins}
  \tableofcontents
\end{frame}

\section{Descrierea problemei}
\frame{\tableofcontents[currentsection]}

\begin{frame}{Privirea în ansamblu a problemei}
   Trei elemente principale:
   \begin{itemize}
  \item Utilizatorul
  \item Jocul video
  \item Recenzia
  \end{itemize}
\end{frame}

\begin{frame}{Specificații}
    Fie  \(U \)  mulțimea de utilizatori și \( G \) mulțimea de  jocuri. 
     \[ \forall  u  \in  U,  \;  \exists \; G_u \subseteq  G  \] 
     
     \[ \forall g \in G_u, \; s(u, g) = r_{ug}\]
     
     \[ 1 \leq r_{ug} \leq 10\]
     
    
    
\end{frame}

\begin{frame}{Obiectiv}
\begin{itemize}
  \item Fie \( u \in U \), \(g \notin G_u \) 
  \item \( \hat{s}(u, g) = ?\)
  \end{itemize}
\end{frame}

\section{Neo4j}
\frame{\tableofcontents[currentsection]}
\begin{frame}{Introducere în Neo4j}
    test
\end{frame}

\begin{frame}{Limbajul de interogare Cypher}
    test2
\end{frame}

\begin{frame}{Cypher vs. SQL}
    test2
\end{frame}

\begin{frame}{Schema bazei de date}
    test2
\end{frame}

\section{Algoritmul de recomandare}
%Contents with current section highlighted
\frame{\tableofcontents[currentsection]}
\begin{frame}{Second section slide}
    test3
\end{frame}

\section{Concluzii}
\frame{\tableofcontents[currentsection]}
\begin{frame}{Concluzii}
	concluzie
\end{frame}

\end{document}