\documentclass[12pt,a4paper]{report}

\usepackage{graphicx}
\usepackage{ragged2e}
\usepackage{geometry}
\usepackage{setspace}
\usepackage{comment}
\usepackage{biblatex}
\usepackage{hyperref}
\usepackage{fancyvrb}
\usepackage{caption}
\usepackage{float}
\addbibresource{sample.bib}

\geometry{
 a4paper,
 total={170mm,257mm},
 left=25mm,
 right=25mm,
 top=20mm,
 }

\newcommand\myrepeat[2]{%
  \begingroup
  \lccode`m=`#2\relax
  \lowercase\expandafter{\romannumeral#1000}%
  \endgroup
}
\linespread{1.213}
\begin{document}
\begin{comment}Copertă 
\end{comment}
\begin{titlepage}
	\centering
	{\scshape\large \textbf{UNIVERSITATEA "ALEXANDRU IOAN CUZA" DIN IAȘI} \par}
	\vspace{0.5cm}
	{\scshape\Large \textbf{FACULTATEA DE INFORMATICĂ}\par}
	\vspace{2cm}
	\includegraphics[width=0.15\textwidth]{logo-fii}\par\vspace{1cm}
	{\scshape\normalsize LUCRARE DE LICENȚĂ\par}
	\vspace{1cm}
	{\huge\bfseries Gemixque\par}
	\vspace{1cm}
	{\normalsize \textbf{sistem de recomandări de jocuri video}\par}
	\vspace{1.5cm}
	{\normalsize \textbf{propusă de}\par}
	\vspace{1cm}
	{\Large\itshape\textbf {Radu Damian}\par}
	\vspace{2cm}
	{\normalsize \textbf{Sesiunea:} \textit{iunie/iulie, 2022}\par}
	\vspace{1cm}
	{\normalsize \textbf{Coordonator științific}\par}
	\vspace{0.5cm}
	{\Large\itshape\textbf {Lect.dr. Cristian Frăsinaru}\par}
\end{titlepage}

\begin{comment}Prima pagină
\end{comment} 
\begin{titlepage}
	\centering
	{\scshape\large \textbf{UNIVERSITATEA "ALEXANDRU IOAN CUZA" DIN IAȘI} \par}
	\vspace{0.5cm}
	{\scshape\Large \textbf{FACULTATEA DE INFORMATICĂ}\par}
	\vspace{5cm}


	{\huge\bfseries Gemixque\par}
	\vspace{1cm}
	{\normalsize \textbf{sistem de recomandări de jocuri video}\par}
	\vspace{3cm}
	{\Large\itshape\textbf {Radu Damian}\par}
	\vspace{2cm}
	{\normalsize \textbf{Sesiunea:} \textit{iunie/iulie, 2022}\par}
	\vspace{4cm}
	{\normalsize \textbf{Coordonator științific}\par}
	\vspace{0.5cm}
	{\Large\itshape\textbf {Lect.dr. Cristian Frăsinaru}\par}
\end{titlepage}

\begin{comment}Declarație standard privind originalitatea lucrării
\end{comment}
\begin{titlepage} 


	\begin{FlushRight}Avizat,\end{FlushRight}
	\begin{FlushRight}Îndrumător Lucrare de Licență				\end{FlushRight}
	\begin{FlushRight}Titlul, Numele și Prenume \hspace{2em}	\rule{5cm}{0.15mm}
	\end{FlushRight}

	\begin{FlushRight}Data \rule{2cm}{0.15mm} \hspace{2em} 	Semnătura \rule{2cm}{0.15mm} \end{FlushRight}

	\hfill \break

	\begin{center}
	\textbf{DECLARAȚIE privind originalitatea conținutului 	lucrării de licență}
	\end{center}
	
	
	\justifying
	\hspace{1em} Subsemnatul(a) \hspace{1em} \myrepeat{90}{.} \newline
	domiciliul în \myrepeat{90}{.} \newline
	născut(ă) la data de \myrepeat{30}{.},\hspace{1em} identificat prin CNP 
\myrepeat{40}{.},
	absolvent(a) al(a) Universității "Alexandru Ioan Cuza" din Iași, \newline
	 Facultatea de \myrepeat{30}{.} \hspace{1em} specializarea \hspace{1em} \myrepeat{55}{.},
	 promoția \hspace{1em} \myrepeat{30}{.}, declar pe propria
	 răspundere, cunoscând consecințele \newline falsului în
	 declarații în sensul art. 326 din Noul Cod Penal și 
	 dispozițiile Legii Educației Naționale nr. 1/2011
	 art.143 al. 4 si 5 referitoare la plagiat, că lucrarea de
	 licență cu titlul: \rule{16cm}{0.15mm} \newline
	 \rule{16cm}{0.15mm} \newline
	 elaborată sub îndrumarea dl. / d-na \rule{9.3cm}{0.15mm}
	 \newline pe care urmează să o susțină în fața
	 comisiei este originală, îmi aparține și îmi asum 				 conținutul său în întregime. 
	 
	 \hspace{1em} De asemenea, declar că sunt de acord ca
	 lucrarea mea de licență să fie verificată prin orice
	 modalitate legală pentru confirmarea originalității,
	 consimțind inclusiv la introducerea conținutului său
	 într-o bază de date în acest scop.
	 
	 \hspace{1em} Am luat la cunoștință despre faptul că este interzisă
	 comercializarea de lucrări științifice în vederea
	 facilitării falsificării de către cumpărător a calității
	 de autor al unei lucrări de licență, de diplomă sau de
	 disertație în acest sens, declar pe proprie răspundere
	 că lucrarea de față nu a fost copiată ci reprezintă rodul 
	 cercetării i pe care am întreprins-o.
	 
	 \hfill \break
	 
	 \hspace{1em} Dată azi, \myrepeat{30}{.}
	 \hspace{5em} Semnătură student \myrepeat{30}{.}
	 
\end{titlepage}

\begin{comment}Declarație standard privind drepturile de utilizare a lucrării și a codului sursă
\end{comment}
\begin{titlepage}
	\hfill \break
	\begin{center}
	DECLARAȚIE DE CONSIMȚĂMÂNT
	\end{center}
	
	\hfill \break
	\justifying
	Prin prezenta declar că sunt de acord ca Lucrarea de
	licență cu titlul \emph{"Gemixque"},
	codul sursă al programelor și celelalte conținuturi
	(grafice, multimedia, date de test etc.) care însoțesc
	această lucrare să fie utilizate în cadrul Facultății
	de Informatică.
	
	De asemenea, sunt de acord ca Facultatea de Informatică
	de la Universitatea "Alexandru Ioan Cuza" din Iași,
	să utilizeze, modifice, reproducă și să distribuie în
	scopuri necomerciale programele-calculator, format
	executabil și sursă, realizate de mine în cadrul prezentei
	lucrări de licență.
	
	\hfill \break
	\hfill \break
	\hfill \break
	\hfill \break
	\hfill \break
	Iași, \emph{data}
	
	\hfill \break
	\hfill \break
	\hfill \break
	\hfill \break
	\hfill \break
	\begin{FlushRight}Absolvent  \emph{Radu Damian} \end{FlushRight}
	\begin{FlushRight}\rule{4.5cm}{0.15mm} \end{FlushRight}
	\begin{FlushRight}(semnătura în original) \end{FlushRight}
	
	
\end{titlepage}

\hfill \break
	\begin{center}
	{\scshape\large \textbf{Mulțumiri} \par}
	\end{center}
	
\begin{flushright}
\hfill \break
\hfill \break
Sunt profund recunoscător pentru sprijinul necondiționat oferit de părinții și de bunica mea în vederea realizării acestei lucrări de licență.
\end{flushright}
\hfill \break
\hfill \break
\hfill \break
\begin{flushright}
Îi mulțumesc Monicăi pentru faptul că m-a ajutat să-mi gestionez trăirile și să privesc provocările pe care le-am întâmpinat cu o atitudine sănătoasă.
\hfill \break
\hfill \break
\hfill \break
\end{flushright}
\begin{flushright}
De asemenea, îi mulțumesc domnului prof. dr Frăsinaru Cristian, îndrumător al lucrării de licență, pentru sfaturile, resursele, și încrederea oferită de-a lungul perioadei în care am lucrat pentru această lucrare.
\end{flushright}

\newpage
   
\renewcommand*{\contentsname}{Cuprins}
\tableofcontents
\newpage

\hfill \break

	\hfill \break
	\begin{center}
	{\scshape\large \textbf{Introducere} \par}
	\end{center}
	
	Această lucrare de licență propune prezentarea implementării unui sistem de recomandare, sub forma unei aplicații web.
	
	În următoarele capitole vor fi ilustrate următoarele: problema ce trebuie rezolvată, tehnologiile folosite, modelarea bazei de date, modul în care au fost generate datele de intrare ale problemei, server-ul de back-end, algoritmul implementat și server-ul de front-end.
	
	\begin{center}
	{\scshape\large \textbf{Motivație} \par}
	\end{center}

	Ideea de bază a lucrării de licență a venit oarecum din întâmplare. Când eram student în semestrul II din anul 2 de facultate, mă uitasem într-o zi obișnuită la acest site: \href{https://simkl.com/}{https://simkl.com}, pe care-l foloseam pentru a nu uita la ce episod am rămas dintr-un anumit serial. Acest site poate oferi și recomandări de filme/seriale în funcție de datele din profilul meu.
	
	Fiind un amator de jocuri video pe PC, mi-a venit ideea de a face un astfel de site, însă care să recomande, bineînțeles, jocuri video.
	
	O observație importantă este faptul că aș fi putut alege orice resursă ce poate fi recomandată(de exemplu: periuțe de dinți, telefoane, picturi etc.), așadar problema în speță poate fi privită într-un mod mai general. Motivul pentru care am ales jocurile video este dat de pasiunea mea pentru acestea.
	
	De asemenea, eram curios în acea perioadă, printre altele, legat de cum aș putea implementa o soluție pentru acest sistem de recomandare pe care mi l-am propus să-l realizez.
	
	Așadar, am avut inspirație, dar și noroc, deoarece această idee mi-a venit relativ repede, lucru ce a reprezentat un punct de pornire important în realizarea efectivă a acestei lucrări.
	
	\newpage
	
\setcounter{secnumdepth}{3} 
\renewcommand*\thesection{\arabic{section}}
\section{Descrierea problemei}

Problema ce trebuie rezolvată se rezumă la recomandarea unor jocuri video(niciunul, unul, sau mai multe) unui utilizator. În ansamblu, datele de intrare ar putea fi reprezentate de o mulțime de jocuri video, o mulțime de utilizatori și o mulțime de recenzii.

Aceste recenzii sunt oferite de utilizator și asociate unui joc. În recenzie, utilizatorul oferă o notă(un scor, pe o scară de la 1 la 10) care indică cât de mult i-a plăcut/displăcut jocul respectiv. Un utilizator poate face maxim o recenzie per joc.
\section{Tehnologii șî resurse folosite}

Pentru aplicația web:

\begin{enumerate}
  \item Baza de date: Neo4j.
  \item Server de back-end:
  \begin{itemize}
     \item Limbaj: Java versiunea 11
     \item Framework: Spring Boot
     \item REST API
   \end{itemize}
  \item Server de front-end:
  \begin{itemize}
     \item Limbaj: Typescript
     \item Framework: Angular
   \end{itemize}
\end{enumerate}
Pentru modulul de generare al datelor de intrare:

\begin{enumerate}
  \item API extern folosit pentru a procura date despre jocuri video: IGDB API \cite{1}
  \item API extern folosit pentru procurarea de recenzii de pe Steam \cite{2}
  \item Java Faker - pentru generarea de date aleatoare \cite{3}
  \item Apache Commons CSV - pentru manipularea de fișiere CSV \cite{4}
\end{enumerate}


\section{Scopuri și cerințe ale aplicației}
\subsection{Scopul documentului}
Scopul acestui document este de a ilustra modul în care funcționează un sistem de recomandări de jocuri video sub forma unei rețele sociale.
\subsection{Publicul țintă}
Acest document este destinat atât cititorilor avizați(e.g. profesori universitari) pentru a afla de exemplu soluțiile utilizate în cadrul capitolului 4 \textbf{Analiză și proiectare} , sau modul de implementare a aplicației în capitolul 5 \textbf{Implementare}, cât și cititorilor neavizați(utilizatori obișnuiți), care se pot informa în legătură cu modul în care pot interacționa cu site-ul în capitolul 6 \textbf{Manual de utilizare}.
\subsection{Scopul aplicației}
Scopul acestei aplicații este de a oferi o soluție în a contracara cantitatea masivă de informații ce se regăsește pe internet \cite{5} în ceea ce privește multitudinea de jocuri video, prin a dezvolta un sistem de recomandări care să faciliteze decizia unui utilizator legat de ce joc să aleagă.  




\section{Analiză și proiectare}
\subsection{Baza de date}

Tipul de stocare ales este cel oferit de Neo4j, în care se prezintă o abordare NoSQL de tip graf.  Comparativ cu o bază de date relațională, în care datele sunt stocate prin intermediul unor înregistrări(tuple) în tabele, în Neo4j datele sunt stocate prin intermediul nodurilor și muchiilor.

În exemplul următor, vor fi ilustrate caracteristicile nodurilor și muchiilor în stocarea efectivă a datelor:


\begin{figure}[h]
\centering
\caption{}
\includegraphics[scale=0.5]{exemplu_1_neo4j}
\caption*{}
\end{figure}


Un nod în neo4j are următoarele caracteristici \cite{6}: reprezintă entități/obiecte, pot fi etichetate și pot avea proprietăți.

Observăm din figură cele trei entități: două noduri etichate cu 'Actor' și un nod etichetat cu 'Movie'. De asemenea, ambele noduri 'Actor' au proprietatea 'name', însă doar unul din ele are și proprietatea 'age'. Așadar, nu trebuie neapărat ca două noduri cu aceeași etichetă să aibă aceleași proprietăți.

Încă un lucru important de menționat este faptul că un nod poate avea mai multe etichete, așa cum se poate vedea din următorul exemplu:


\begin{figure}[h]
\centering
\caption{}
\includegraphics[scale=0.5]{exemplu_2_neo4j}
\caption*{}
\end{figure}


Practic, din această figură se observă faptul că se poate modela cu ușurință situația în care un regizor joacă în propriul său film.

O altă noțiune importantă într-o bază de date de tip graf este cea de relație, care este asociată unei muchii. 
O relație trebuie să aibă un tip, un sens, și poate avea proprietăți.
Din figurile anterioare s-au putut observa relațiile \texttt{ACTS\_IN}, \texttt{DIRECTS\_IN} sau \texttt{IS\_FRIEND}.

Un alt aspect important de precizat este faptul că între două entități pot exista mai multe relații, așa cum s-a putut vedea în figura anterioară. Așadar, se modelează practic un multigraf orientat.

\subsection{De ce NoSQL și nu SQL?}

Având în vedere faptul că în această aplicație este prezentă o rețea socială în care mai mulți utilizatori pot interacționa între ei și pot oferi recenzii jocurilor video, o bază de date de tip graf este o alegere inspirată.

În ceea ce privește limbajul utilizat pentru a efectua interogări pe baza de date, avem în vedere următorul studiu de caz:

Să presupunem că dorim să modelăm ceea ce se întâmplă în cadrul unei facultăți, pe scurt: gestionarea studenților, a notelor pe care le iau aceștia la cursuri, și a profesorilor.
Exemplul ce urmează a fi ilustrat se bazează pe schema bazei de date ce a fost utilizată în cadrul materiei Baze de date din anul II semestrul I. \cite{7}

Să propunem că dorim să efectuăm următoarea interogare: pentru un student, să aflăm numele profesorilor la cursurile în care studentul a luat nota 10.

O interogare în limbajul SQL ar putea arăta în felul următor:

\begin{figure}[h]
\centering
\begin{BVerbatim}
SELECT p.nume, p.prenume FROM NOTE n 
JOIN CURSURI c ON n.id_curs = c.id
JOIN DIDACTIC d ON d.id_curs = c.id
JOIN PROFESORI p ON p.id = d.id_profesor
WHERE VALOARE = 10 AND ID_STUDENT = 36;
\end{BVerbatim}
\end{figure}


Se poate observa așadar faptul că sunt necesare o serie de join-uri pentru a putea obține rezultatul dorit.


Pentru a putea compara această interogare cu cea care s-ar putea face în limbajul Cypher folosit în Neo4j, se va ilustra pe scurt cum s-ar putea modela schema bazei de date anterior menționată în una de tip graf. (nu în totalitate, ci doar de ceea ce avem nevoie pentru a evidenția interogarea)


\begin{figure}[h]
\centering
\caption{Miniatură a schemei bazei de date}
\includegraphics[scale=0.4]{exemplu_3_neo4j}
\caption*{}
\end{figure}


Așadar, având în vedere acest model, interogarea în Cypher ar putea arăta în felul următor:

\begin{figure}[h]
\centering
\begin{BVerbatim}
MATCH (s:STUDENT)-[:ARE]->(n:NOTA {valoare: 10}),
(n)-[:LA]->(:CURS)<-[:PREDA]-(p:PROFESOR)
WHERE id(s) = 0
RETURN p.nume, p.prenume
\end{BVerbatim}
\end{figure}



Un prim lucru interesant care s-ar putea observa este faptul că Cypher introduce prin sintaxa sa conceptul de ASCII Art \cite{8}, \cite{9}, care practic face posibilă o nouă interpretare a codului, una vizuală, adică se poate observa relativ ușor cum ceea ce este scris în primele două linii de cod seamănă destul de mult cu modelul din figura anterioară.
Cele două linii s-ar fi putut scrie într-o singură linie fără nicio dificultate, însă am optat pentru această variantă pentru o mai bună lizibilitate a codului.

Pe de altă parte, o altă caracteristică importantă a bazelor de date de tip graf este că relațiile între entități au cea mai mare prioritate în modelarea datelor \cite{10}, acest lucru reflectându-se în modul cum a fost construită această interogare, în care este mult mai facil să interogăm niște date aflate la capătul unui lanț format din mai multe relații, așa cum se observă și în acest exemplu.

Pentru o mai bună înțelegere a modului în care sunt reprezentate efectiv datele, se poate vedea figura de mai jos:

\begin{figure}[h]
\centering
\caption{}
\includegraphics[scale=0.5]{exemplu_4_neo4j}
\caption*{Datele folosite pentru a exemplifica interogarea în Cypher}
\end{figure}

\subsection{Modelarea bazei de date}


Având în vedere noțiunile prezentate în capitolele anterioare, în acest capitol va fi descrisă schema bazei de date ce va fi folosită pentru a stoca datele necesare rezolvării problemei de a recomanda jocuri video.

Însă, înainte de a ilustra schema în ansamblu, fiecare entitate în parte va fi detaliată în rândurile următoare.

\bigskip
\textbf{Utilizatorul}
\bigskip

Această entitate reprezintă punctul central al aplicației, deoarece utilizatorul este cel care oferă recenzii(note) jocurilor și influențează într-un mod indirect recomandările unui alt utilizator, deci el este cel care, practic, inițiază acțiunea.

Nodul corespunzător acestei entități este cel din \textbf{Figura 5}.

Printre proprietățile alese pentru a reprezenta un utilizator, două dintre acestea reflectă activitatea lui în sistemul de recomandare, anume \emph{average\underline{ }score} și \emph{nr\underline{ }reviews\underline{ }made}, care semnifică numărul total de recenzii făcute de utilizator, respectiv scorul mediu al acestuia(media tuturor scorurilor oferite).

\bigskip
\textbf{Recenzia}
\bigskip

Recenzia reprezintă practic indicatorul calitativ pe care-l oferă un utilizator asupra unui joc. Așa cum se observă din 
\textbf{Figura 6}, o recenzie poate fi caracterizată prin proprietatea \emph{score}, nota oferită de utilizator jocului, proprietatea \emph{content}, care reprezintă un text ce poate fi introdus de utilizator pentru a-și exprima, dacă dorește, opinia sa asupra jocului, și proprietatea \emph{time}, adică momentul în care a făcut recenzia.

\bigskip
\textbf{Jocul}
\bigskip

Jocul indică resursa ce va fi recomandată de către utilizator și este punctul de interes al aplicației. Faptul că această entitate poate avea un număr relativ semnificativ de proprietăți reprezintă o provocare justificată, iar abordarea mea pentru a reprezenta această entitate se poate observa în 
\textbf{Figura 7}.


Pentru a nu suprasatura un nod cu multe proprietăți, am decis să împart acestea în funcție de categoria în care ar putea fi încadrate. Nodul principal \textbf{Game} are doar titlul, genurile și anul primei lansări, iar prin relația de tipul \texttt{HAS} se evidențiază nodurile care extind această entitate, și anume cel care conține proprietățile care indică elemente vizuale ale jocului(\emph{Visuals}), cel care conține date suplimentare despre joc(\emph{Details}) și cel care ține evidența legat de recenziile aplicate asupra jocului (\emph{Average}), unde proprietatea \emph{value} reprezintă scorul mediu al jocului(cât de bine este notat în medie de către utilizatorii sistemului), iar proprietatea \emph{aggregated\_rating} reprezintă scorul obținut dintr-o sursă externă(e.g. Metacritic)

\bigskip
\textbf{Schema bazei de date}
\bigskip

În \textbf{Figura 8}, având în vedere și aspectele menționate anterior legat de entitățile care formează în ansamblu schema bazei de date, se poate intui fluxul principal al aplicației. Utilizatorul introduce în profilul său ce jocuri video s-a jucat prin intermediul relației \texttt{PLAYS}, iar acțiunea prin care acesta efectuează o recenzie asupra jocului este reprezentată prin relația de tip \texttt{MAKES}. De asemenea, relația \texttt{ON} indică asupră cărei entități etichetată cu \emph{Game} se face recenzia.

Se poate deduce faptul că relațiile între noduri sunt numite astfel încât să reprezinte acțiuni ce pot fi efectuate sau aplicate asupra unor entități, ceea ce poate reprezenta un procedeu de bună practică în modelarea bazei de date în acest context. \cite{11}



\begin{figure}[H]
\centering
\caption{Utilizatorul}
\includegraphics[]{exemplu_6_neo4j}
\caption*{}
\end{figure}

\begin{figure}[H]
\centering
\caption{Recenzia}
\includegraphics[]{exemplu_7_neo4j}
\caption*{}
\end{figure}

\begin{figure}[H]
\centering
\caption{Jocul}
\includegraphics[scale = 0.6]{exemplu_8_neo4j}
\caption*{}
\end{figure}



\begin{figure}[H]
\centering
\caption{Schema bazei de date}
\includegraphics[scale=0.4]{exemplu_5_neo4j}
\caption*{}
\end{figure}



\section{Implementare}
\section{Manual de utilizare}
\section{Concluzii}

\renewcommand\bibname{Bibliografie}
\begin{thebibliography}{9}

\bibitem{igdb-api}
  
  \textit{IGDB API}
  
  \url{https://www.igdb.com/api}
  
  \url{https://github.com/husnjak/IGDB-API-JVM}
  
\bibitem{steam-reviews}
  
  \textit{Steamworks Documentation - Reviews}
  
  \url{https://partner.steamgames.com/doc/store/getreviews}
  
\bibitem{java-faker}
  
  \textit{Java Faker}
  
  \url{https://github.com/DiUS/java-faker}
  
\bibitem{apache-commons-csv}
  
  \textit{Apache Commons CSV}
  
  \url{https://commons.apache.org/proper/commons-csv/}


\bibitem{lamport94}
  
  \textit{Recommendation systems: Principles, methods and evaluation},
  2015.
  
  F.O. Isinkaye, Y.O. Folajimi, B.A. Ojokoh
  
  \url{https://www.sciencedirect.com/science/article/pii/S1110866515000341}
  
  \bibitem{graphDatabase}
  
  \textit{What is a graph database?}
  
  \url{https://neo4j.com/developer/graph-database/}
  
  \bibitem{7}
  
  \textit{Baze de date - Laborator 1}
  
  A se vedea secțiunea 'Schema bazei de date - studiu de caz'
  
  
  \url{https://profs.info.uaic.ro/~bd/wiki/index.php/Laborator_1}
  
  \bibitem{8}
  
  \textit{Neo4j Training - Querying with Cypher In Neo4j 4.x}
  
  \url{https://neo4j.com/graphacademy/training-querying-40/01-querying40-introduction-to-cypher/}
  
  \bibitem{9}
  
  \textit{ASCII art}
  
  \url{https://en.wikipedia.org/wiki/ASCII_art}
  
  \bibitem{10}
  
  \textit{Neo4j Training - Overview of Neo4j 4.x - Neo4j is a Graph Database  }
  
  \url{https://neo4j.com/graphacademy/training-overview-40/01-overview40-neo4j-graph-database/}
  
  \bibitem{11}
  
  \textit{Graph Modeling Guidelines}
  
  \url{https://neo4j.com/developer/guide-data-modeling/}

\end{thebibliography}

\end{document}
