\documentclass[12pt,a4paper]{report}

\usepackage{graphicx}
\usepackage{ragged2e}
\usepackage{geometry}
\usepackage{setspace}
\usepackage{comment}
\usepackage{biblatex}
\usepackage{hyperref}
\addbibresource{sample.bib}

\geometry{
 a4paper,
 total={170mm,257mm},
 left=25mm,
 right=25mm,
 top=20mm,
 }

\newcommand\myrepeat[2]{%
  \begingroup
  \lccode`m=`#2\relax
  \lowercase\expandafter{\romannumeral#1000}%
  \endgroup
}
\linespread{1.213}
\begin{document}
\begin{comment}Copertă 
\end{comment}
\begin{titlepage}
	\centering
	{\scshape\large \textbf{UNIVERSITATEA "ALEXANDRU IOAN CUZA" DIN IAȘI} \par}
	\vspace{0.5cm}
	{\scshape\Large \textbf{FACULTATEA DE INFORMATICĂ}\par}
	\vspace{2cm}
	\includegraphics[width=0.15\textwidth]{logo-fii}\par\vspace{1cm}
	{\scshape\normalsize LUCRARE DE LICENȚĂ\par}
	\vspace{1cm}
	{\huge\bfseries Gemixque\par}
	\vspace{1.5cm}
	{\normalsize \textbf{propusă de}\par}
	\vspace{1cm}
	{\Large\itshape\textbf {Radu Damian}\par}
	\vspace{2cm}
	{\normalsize \textbf{Sesiunea:} \textit{luna, 2022}\par}
	\vspace{1cm}
	{\normalsize \textbf{Coordonator științific}\par}
	\vspace{0.5cm}
	{\Large\itshape\textbf {Lect.dr. Cristian Frăsinaru}\par}
\end{titlepage}

\begin{comment}Prima pagină
\end{comment} 
\begin{titlepage}
	\centering
	{\scshape\large \textbf{UNIVERSITATEA "ALEXANDRU IOAN CUZA" DIN IAȘI} \par}
	\vspace{0.5cm}
	{\scshape\Large \textbf{FACULTATEA DE INFORMATICĂ}\par}
	\vspace{5cm}


	{\huge\bfseries Gemixque\par}
	\vspace{3cm}
	{\Large\itshape\textbf {Radu Damian}\par}
	\vspace{2cm}
	{\normalsize \textbf{Sesiunea:} \textit{luna, 2022}\par}
	\vspace{4cm}
	{\normalsize \textbf{Coordonator științific}\par}
	\vspace{0.5cm}
	{\Large\itshape\textbf {Lect.dr. Cristian Frăsinaru}\par}
\end{titlepage}

\begin{comment}Declarație standard privind originalitatea lucrării
\end{comment}
\begin{titlepage} 


	\begin{FlushRight}Avizat,\end{FlushRight}
	\begin{FlushRight}Îndrumător Lucrare de Licență				\end{FlushRight}
	\begin{FlushRight}Titlul, Numele și Prenume \hspace{2em}	\rule{5cm}{0.15mm}
	\end{FlushRight}

	\begin{FlushRight}Data \rule{2cm}{0.15mm} \hspace{2em} 	Semnătura \rule{2cm}{0.15mm} \end{FlushRight}

	\hfill \break

	\begin{center}
	\textbf{DECLARAȚIE privind originalitatea conținutului 	lucrării de licență}
	\end{center}
	
	
	\justifying
	\hspace{1em} Subsemnatul(a) \hspace{1em} \myrepeat{90}{.} \newline
	domiciliul în \myrepeat{90}{.} \newline
	născut(ă) la data de \myrepeat{30}{.},\hspace{1em} identificat prin CNP 
\myrepeat{40}{.},
	absolvent(a) al(a) Universității "Alexandru Ioan Cuza" din Iași, \newline
	 Facultatea de \myrepeat{30}{.} \hspace{1em} specializarea \hspace{1em} \myrepeat{55}{.},
	 promoția \hspace{1em} \myrepeat{30}{.}, declar pe propria
	 răspundere, cunoscând consecințele \newline falsului în
	 declarații în sensul art. 326 din Noul Cod Penal și 
	 dispozițiile Legii Educației Naționale nr. 1/2011
	 art.143 al. 4 si 5 referitoare la plagiat, că lucrarea de
	 licență cu titlul: \rule{16cm}{0.15mm} \newline
	 \rule{16cm}{0.15mm} \newline
	 elaborată sub îndrumarea dl. / d-na \rule{9.3cm}{0.15mm}
	 \newline pe care urmează să o susțină în fața
	 comisiei este originală, îmi aparține și îmi asum 				 conținutul său în întregime. 
	 
	 \hspace{1em} De asemenea, declar că sunt de acord ca
	 lucrarea mea de licență să fie verificată prin orice
	 modalitate legală pentru confirmarea originalității,
	 consimțind inclusiv la introducerea conținutului său
	 într-o bază de date în acest scop.
	 
	 \hspace{1em} Am luat la cunoștință despre faptul că este interzisă
	 comercializarea de lucrări științifice în vederea
	 facilitării falsificării de către cumpărător a calității
	 de autor al unei lucrări de licență, de diplomă sau de
	 disertație în acest sens, declar pe proprie răspundere
	 că lucrarea de față nu a fost copiată ci reprezintă rodul 
	 cercetării i pe care am întreprins-o.
	 
	 \hfill \break
	 
	 \hspace{1em} Dată azi, \myrepeat{30}{.}
	 \hspace{5em} Semnătură student \myrepeat{30}{.}
	 
\end{titlepage}

\begin{comment}Declarație standard privind drepturile de utilizare a lucrării și a codului sursă
\end{comment}
\begin{titlepage}
	\hfill \break
	\begin{center}
	DECLARAȚIE DE CONSIMȚĂMÂNT
	\end{center}
	
	\hfill \break
	\justifying
	Prin prezenta declar că sunt de acord ca Lucrarea de
	licență cu titlul \emph{"Gemixque"},
	codul sursă al programelor și celelalte conținuturi
	(grafice, multimedia, date de test etc.) care însoțesc
	această lucrare să fie utilizate în cadrul Facultății
	de Informatică.
	
	De asemenea, sunt de acord ca Facultatea de Informatică
	de la Universitatea "Alexandru Ioan Cuza" din Iași,
	să utilizeze, modifice, reproducă și să distribuie în
	scopuri necomerciale programele-calculator, format
	executabil și sursă, realizate de mine în cadrul prezentei
	lucrări de licență.
	
	\hfill \break
	\hfill \break
	\hfill \break
	\hfill \break
	\hfill \break
	Iași, \emph{data}
	
	\hfill \break
	\hfill \break
	\hfill \break
	\hfill \break
	\hfill \break
	\begin{FlushRight}Absolvent  \emph{Radu Damian} \end{FlushRight}
	\begin{FlushRight}\rule{4.5cm}{0.15mm} \end{FlushRight}
	\begin{FlushRight}(semnătura în original) \end{FlushRight}
	
	
\end{titlepage}

\begin{titlepage}

	\hfill \break
	\begin{center}
	ACORD PRIVIND PROPRIETATEA DREPTULUI DE AUTOR
	\end{center}
	
	\hfill \break
	\hfill \break
	\justifying
	Facultatea de Informatică este de acord ca drepturile
	de autor asupra programelor-calculator, în format
	executabil și sursă, să aparțină autorului prezentei
	lucrări, \emph{Radu Damian.}
	\hfill \break
	\hfill \break
	Încheierea acestui acord este necesară din următoarele
	motive:
	\hfill \break
	\hfill \break
	\hfill \break
	\emph{...de completat...}
	\hfill \break
	\hfill \break
	\hfill \break
	\hfill \break
	\hfill \break
	\hfill \break
	Iași, \emph{data}
	\hfill \break
	\hfill \break
	\hfill \break
	Decan \emph{Adrian Iftene} \hspace{18em} Absolvent \emph{Radu Damian}  
	\hfill \break
	\hfill \break
	\rule{4.5cm}{0.15mm} \hspace{16em} \rule{4.5cm}{0.15mm}
	(semnătura în original) \hspace{17em} (semnătura în original)
	
	
	

\end{titlepage}   
\renewcommand*{\contentsname}{Cuprins}
\tableofcontents
\newpage

	\hfill \break
	\begin{center}
	{\scshape\large \textbf{Introducere} \par}
	\end{center}
	
	Această aplicație web constă într-un sistem de recomandări de jocuri video încorporat într-o rețea socială.
	
	-punere în cunoștință de cauză a cititorului
	
	-plasarea tematicii în context
	
	-prezentarea clară a scopului și modului de realizare(soluția teoretică/tehnologică oferită)
	
	-punctarea celor mai interesante aspecte + contribuții personale
	
	-prezentarea structurii lucrării( capitolul X se referă...)
	
	-mulțumirile aduse
	
	\newpage
	
\setcounter{secnumdepth}{3} 
\renewcommand*\thesection{\arabic{section}}
\section{Descrierea problemei}
\section{Fundamente}
\section{Scopuri și cerințe ale aplicației}
\subsection{Introducere}
\subsubsection{Scopul documentului}
Scopul acestui document este de a ilustra într-o manieră interactivă sub forma unei rețele sociale(din perspectiva utilizatorului) modul în care funcționează un sistem de recomandări. Domeniul ales este cel al jocurilor video.
\subsubsection{Publicul țintă}
Acest document este destinat atât cititorilor avizați(e.g. profesori universitari) pentru a afla de exemplu soluțiile utilizate în cadrul capitolului 4 \textbf{Analiză și proiectare} , sau modul de implementare a aplicației în capitolul 5 \textbf{Implementare}, cât și cititorilor neavizați(utilizatori obișnuiți), care se pot informa în legătură cu modul în care pot interacționa cu situl în capitolul 6 \textbf{Manual de utilizare}.
\subsubsection{Scopul proiectului}
Scopul acestui proiect este de a oferi o soluție în a contracara cantitatea masivă de informații ce se regăsește pe internet \cite{1} în ceea ce privește multitudinea de jocuri video, prin a dezvolta un sistem de recomandări care să faciliteze decizia unui utilizator legat de ce joc să aleagă.  
\section{Analiză și proiectare}
\subsection{Baza de date}
\section{Implementare}
\section{Manual de utilizare}
\section{Concluzii}

\renewcommand\bibname{Bibliografie}
\begin{thebibliography}{9}

\bibitem{lamport94}
  F.O. Isinkaye, Y.O. Folajimi, B.A. Ojokoh
  
  \textit{Recommendation systems: Principles, methods and evaluation},
  2015.
  
  \url{https://www.sciencedirect.com/science/article/pii/S1110866515000341}

\end{thebibliography}

\end{document}
